%==============================================================================
% Sjabloon onderzoeksvoorstel bachelorproef
%==============================================================================
% Gebaseerd op LaTeX-sjabloon ‘Stylish Article’ (zie voorstel.cls)
% Auteur: Jens Buysse, Bert Van Vreckem
%
% Compileren in TeXstudio:
%
% - Zorg dat Biber de bibliografie compileert (en niet Biblatex)
%   Options > Configure > Build > Default Bibliography Tool: "txs:///biber"
% - F5 om te compileren en het resultaat te bekijken.
% - Als de bibliografie niet zichtbaar is, probeer dan F5 - F8 - F5
%   Met F8 compileer je de bibliografie apart.
%
% Als je JabRef gebruikt voor het bijhouden van de bibliografie, zorg dan
% dat je in ``biblatex''-modus opslaat: File > Switch to BibLaTeX mode.

\documentclass{voorstel}

\usepackage{lipsum}

%------------------------------------------------------------------------------
% Metadata over het voorstel
%------------------------------------------------------------------------------

%---------- Titel & auteur ----------------------------------------------------

% TODO: geef werktitel van je eigen voorstel op
\PaperTitle{Handschriftherkenning voor doktersbriefjes}
\PaperType{Onderzoeksvoorstel Bachelorproef 2019-2020} % Type document

% TODO: vul je eigen naam in als auteur, geef ook je emailadres mee!
\Authors{Helmi Dayyeh\textsuperscript{1}} % Authors
\CoPromotor{Wouter Baetens}


%---------- Abstract ----------------------------------------------------------

\Abstract{
Met de opkomst van \textbf{artificiële intelligentie} worden alledaagse taken al vaak geautomatiseerd. Vandaag de dag hebben bedrijven de tijd en het budget niet om talloze mensen aan te werven om terugkerende manuele taken uit te voeren. Bovendien zijn mensen gekend om fouten te maken en kan dit een negatieve invloed op de workflow hebben.  

  

Deze analyse is gebaseerd op een recent onderzoek uitgevoerd door Vectr.Consulting in samenwerking met de Christelijke mutualiteit. De probleemstelling bij dit onderzoek bleek te zijn dat de handgeschreven overlijdensakten niet leesbaar waren. Dit heeft men dan doormiddel van een applicatie die gebruik maakt van het Tenserflow framework met een getraind neuraal netwerk opgelost, waarbij in \textbf{47 \%} van de gevallen het handgeschrift correct voorspeld werd.   

  

De vraag en hetgeen dit onderzoek zal gaan vaststellen is of via deze handtering ook het handgeschrift op doktersbriefjes kan herkend worden.   

  

In de literatuurstudie zal tot in detail de opzet en de statistieken van het neuraal netwerk beschreven staan. Het grootste obstakel zal dan ook uiteraard de data zijn die verzameld zal moeten worden van een derde partij.  

  

Het praktische gedeelte van dit onderzoek zal gebruik maken een app die in de achtergrond \textbf{azure cognitive services} gebruikt om het model te trainen en predicties te maken.   

  

Door dit onderzoek uit te voeren met een praktisch voorbeeld zal de christelijke mutualiteit een meerwaarde zien in de overstap naar   \textbf{artificiële intelligentie} binnen hun organisatie. 
}

%---------- Onderzoeksdomein en sleutelwoorden --------------------------------
% TODO: Sleutelwoorden:
%
% Het eerste sleutelwoord beschrijft het onderzoeksdomein. Je kan kiezen uit
% deze lijst:
%
% - Mobiele applicatieontwikkeling
% - Webapplicatieontwikkeling
% - Applicatieontwikkeling (andere)
% - Systeembeheer
% - Netwerkbeheer
% - Mainframe
% - E-business
% - Databanken en big data
% - Machineleertechnieken en kunstmatige intelligentie
% - Andere (specifieer)
%
% De andere sleutelwoorden zijn vrij te kiezen

\Keywords{Onderzoeksdomein. AI, Machine learning --- Azure Cognitive Services } % Keywords
\newcommand{\keywordname}{Sleutelwoorden} % Defines the keywords heading name

%---------- Titel, inhoud -----------------------------------------------------

\begin{document}

\flushbottom % Makes all text pages the same height
\maketitle % Print the title and abstract box
\tableofcontents % Print the contents section
\thispagestyle{empty} % Removes page numbering from the first page

%------------------------------------------------------------------------------
% Hoofdtekst
%------------------------------------------------------------------------------

% De hoofdtekst van het voorstel zit in een apart bestand, zodat het makkelijk
% kan opgenomen worden in de bijlagen van de bachelorproef zelf.
%---------- Inleiding ---------------------------------------------------------

\section{Introductie} % The \section*{} command stops section numbering
\label{sec:introductie}

\textbf{Probleemstelling:}
Handgeschrift op doktersbriefjes is in sommige gevallen onleesbaar voor de christelijke Mutualiteit. Het is nu ook niet wetenschappelijk bewezen dat dokters het slechtste handgeschrift hebben, maar het wordt wel sociaal vastgesteld. 
\\\textbf{Motivatie:}
Het gebied van Artificiële intelligentie en machine learning spreekt toekomst. Bijna ieder toestel maakt al op een manier gebruik van AI. Door de opkomst van IoT genereren miljoenen gebruikers dagelijks data die gebruikt wordt om de accuraatheid van AI geïntegreerde systemen te vergroten. \newline We kunnen hier dan ook uit vaststellen dat in de nabije toekomst het gebruik van AI alleen maar effectiever zal worden. Het is dus belangrijk dat we hier zo snel mogelijk op inspelen en een domein onderzoeken waar we dit dan ook kunnen gebruiken.  
\\\textbf{Doelstelling:}
De werkelijke onderzoeksvraag is dus of de workflow efficiëntie van de christelijke Mutualiteit verbeterd kan worden door middel van het integreren van handschriftherkenning. Dit zal natuurlijk in meerdere stappen uitgevoerd worden. Door het verzamelen van voorgaande data en het trainen van het neuraal netwerk wordt verwacht dat dit model het handgeschrift op doktersbriefjes kan herkennen/ parsen met een efficiëntie van +- 47 \%. Dit zal uiteraard naarmate de grote van de dataset verbeteren. 



%---------- Stand van zaken ---------------------------------------------------

\section{State-of-the-art}

\label{sec:state-of-the-art}

 
Zoals vooraf al aan bod kwam is deze onderzoeksvraag gebaseerd op een recent onderzoek dat uitgevoerd werd door Vectr.Consulting in samenwerking met de Christelijke Mutualiteit. In dit onderzoek werd een model opgezet en getraind om doktershandgeschrift te herkennen op doodsakten. Verder, baseerde het praktische gedeelte van dit uitgevoerde onderzoek zich op het open source Tenserflow framework. Dit onderzoek zal vooral gebaseerd zijn rond de Microsoft Azure stack, meer bepaald de afdeling Cognitive Services. Microsoft hanteert hiervoor hun krachtig cloud networking solution om op basis van vooraf getrainde modellen realtime data te bezorgen aan de consument. Dit framework is vooral nog in de early development stages en zal naarmate het gebruik alleen maar verbeteren. 

% Voor literatuurverwijzingen zijn er twee belangrijke commando's:
% \autocite{KEY} => (Auteur, jaartal) Gebruik dit als de naam van de auteur
%   geen onderdeel is van de zin.
% \textcite{KEY} => Auteur (jaartal)  Gebruik dit als de auteursnaam wel een
%   functie heeft in de zin (bv. ``Uit onderzoek door Doll & Hill (1954) bleek
%   ...'')



%---------- Methodologie ------------------------------------------------------
\section{Methodologie}
\label{sec:methodologie}
In zeker zin zal er een observatieonderzoek plaats vinden. De dataset zal bestaan uit vooraf geschreven doktersbriefjes waarbij het computer vision gedeelte van azure cognitieve services de data uit elke image extraheert en analyseert. Door deze methodologie zal de eindgebruiker de mogelijkheid hebben de geëxtraheerde data een rating te geven op basis van accuraatheid van foto naar tekst extractie. Het model zal hierdoor alleen maar bijleren en zichzelf aanpassen naar toekomstige data. 

% https://blogs.bing.com/search-quality-insights/2017-07/bring-rich-knowledge-of-people-places-things-and-local-businesses-to-your-apps/

%---------- Verwachte resultaten ----------------------------------------------
\section{Verwachte resultaten}
\label{sec:verwachte_resultaten}
Na het trainen en modelleren van de data zal het handgeschrift op doktersattesten met een nauwkeurigheid van omtrent 47 \% voorspeld kunnen worden. Data zal gestructureerd worden in functie van de kwaliteit en hoe zeker het model is van zijn voorspelling. Om dit percentage te bereiken zal het model enkel getraind worden met data die aan een bepaalde kwaliteitsstandaard voldoet en zal na elke voorspelling de eindgebruiker geprompt worden om de kwaliteit van extractie te raten. Medische termen zullen hierbij een zekere rol spelen, aangezien deze niet altijd de makkelijkste zijn om te herkennen. 


%---------- Verwachte conclusies ----------------------------------------------
\section{Verwachte conclusies}
\label{sec:verwachte_conclusies}
Wanneer de bevindingen concorderen met hetgeen verwacht wordt, wordt er gesteld dat de Christelijke Mutualiteit een meerwaarde in workflow en resources zal bekomen, in de hoop dat meerdere taken door artificiële intelligentie geautomatiseerd zullen worden. Dit is natuurlijk een oplossing die op lange termijn zichzelf zal verbeteren tot op het punt dat de accuraatheid van de voorspelde tekst naar een 80-90 \% convergeert.  

\section{Referenties}
\label{sec:verwachte_conclusies}
https://vectr.consulting/news/deciphering-handwriting
\newline
https://vectr.consulting/news/smart-scanner/artificial-intelligence
\newline
https://azure.microsoft.com/en-us/services/cognitive-services/computer-vision/
\newline
https://towardsdatascience.com/build-a-handwritten-text-recognition-system-using-tensorflow-2326a3487cd5

%------------------------------------------------------------------------------
% Referentielijst
%------------------------------------------------------------------------------
% TODO: de gerefereerde werken moeten in BibTeX-bestand ``voorstel.bib''
% voorkomen. Gebruik JabRef om je bibliografie bij te houden en vergeet niet
% om compatibiliteit met Biber/BibLaTeX aan te zetten (File > Switch to
% BibLaTeX mode)

\phantomsection
\printbibliography[heading=bibintoc]

\end{document}
