%%=============================================================================
%% Voorwoord
%%=============================================================================

\chapter*{\IfLanguageName{dutch}{Woord vooraf}{Preface}}
\label{ch:voorwoord}

%% TODO:
%% Het voorwoord is het enige deel van de bachelorproef waar je vanuit je
%% eigen standpunt (``ik-vorm'') mag schrijven. Je kan hier bv. motiveren
%% waarom jij het onderwerp wil bespreken.
%% Vergeet ook niet te bedanken wie je geholpen/gesteund/... heeft
Deze bachelorscriptie werd geschreven met het oog op het behalen van de titel: Bachelor in de Toegepaste Informatica. 


Naar aanleiding van mijn stage bij De Cronos Groep, leek het de perfecte opportuniteit om in het domein van Cloud Computing en Artificiële intelligentie een onderzoek te starten. Hierbij kwam Arinti (een dochterbedrijf van De Cronos Groep) met een zeer interessant onderzoeksvoorstel.  De Christelijke Mutualiteit had namelijk problemen met het lezen van doktersgeschrift op medische voorschriften. Hierbij kwam de vraag of dit proces eveneens geautomatiseerd kon worden door gebruik te maken van AI, en of er een verschil zou zijn in performantie/accuraatheid.  


Aangezien het voorstel van dit onderzoek mij aangeboden werd vanuit Arinti, wil ik mijn co-promoter, Wouter Baetens bedanken om mij de kans te geven om dit zeer interessant onderzoek uit te voeren en enige vragen te beantwoorden die naar boven kwamen. Ook wil ik mijn stagebegeleider, Geert Baeke, bedanken om mij voor te stellen aan Arinti en zo aanleiding tot dit onderzoek te geven. Verder bedank ik mijn promotor, Irina Malfait, voor de snelle feedback en sturing van de bachelorproef. 

Tot slot wil ik mijn ouders en vrienden bedanken voor de constante steun bij het uitvoeren van dit onderzoek. 